\documentclass[10pt,aspectratio=169]{beamer}

\usepackage[utf8x]{inputenc}
\usepackage[T1]{fontenc}
\usepackage{lmodern}

\usepackage{amsmath,amssymb,mathtools}
\usepackage{bm}
\usepackage{empheq}
\usepackage{graphicx}
\usepackage{array,booktabs,tabularx,tabulary,multirow}
\usepackage{tikz}
\usepackage{hyperref}
\usepackage{feynmp}

\newcolumntype{V}{>{\centering\arraybackslash} m{.2\linewidth} }

\usetikzlibrary{arrows,matrix,shapes,positioning}
\usetikzlibrary{calc}
\usetikzlibrary{shadows.blur}

\newcommand*\widefbox[1]{\fbox{\hspace{0.5em}#1\hspace{0.5em}}}
\newcommand{\DRbar}{{\overline{\textrm{DR}}}}

\DeclareMathOperator{\Tr}{Tr}

\DeclareGraphicsRule{*}{mps}{*}{}
\graphicspath{{./figures/}}

\title{Minimal Supersymmetry and Beyond}

\author{D.~Harries\\
  {\scriptsize
  (IPNP, Charles University in Prague)}
  }

\titlegraphic{
  \begin{center}
    \hspace*{\fill}
    \includegraphics[scale=0.3]{uk_logo}
    \hspace*{\fill}
  \end{center}
}

\date[\'{U}TF, Charles University in Prague]{March 27, 2018}

\usetheme{CambridgeUS}

\setbeamertemplate{headline}[default]{}
\setbeamertemplate{footline}[page number]{}
\setbeamertemplate{navigation symbols}{}

\begin{document}

\begin{frame}[plain]
  \titlepage
\end{frame}

\begin{frame}
  \frametitle{Outline}
  \tableofcontents
\end{frame}

% general aspects of SUSY
\section{General Features}

% very general mention of some empirical reasons for BSM
% physics before specialising to the theoretical motivations
% for studying SUSY as a feature of BSM models
\begin{frame}
  \frametitle{The Case for BSM Physics}
  \begin{columns}[t]
    \begin{column}{0.3\textwidth}
      \vspace*{-15pt}
      \begin{figure}
        \centering
        \includegraphics[width=0.8\textwidth]{bulletcluster}
      \end{figure}
      \vspace{-25pt}
      \begin{center}
        {\tiny [\href{https://apod.nasa.gov/apod/ap060824.html}{APOD/NASA}]} \\
        Dark matter?
      \end{center}
      \vspace{-15pt}
      \begin{figure}
        \centering
        \includegraphics[width=0.8\textwidth]{SM_gauge_rgflow}
      \end{figure}
      \vspace{-25pt}
      \begin{center}
        {\tiny [\href{http://flexiblesusy.hepforge.org/images.html}{%
        http://flexiblesusy.hepforge.org/images.html}]} \\
        Gauge unification?
      \end{center}
    \end{column}
    \begin{column}{0.7\textwidth}
      \begin{columns}[t]
        \begin{column}{0.5\textwidth}
          \vspace*{-30pt}
          \begin{figure}
            \includegraphics[width=0.85\textwidth]{neutrino_masses} \\
          \end{figure}
          \vspace*{-25pt}
          \begin{center}
            {\tiny [\href{https://arxiv.org/abs/1611.01514}{%
              arXiv:1611.01514}]} \\
            Neutrino masses?
          \end{center}
        \end{column}
        \begin{column}{0.5\textwidth}
          \begin{figure}
            \includegraphics[width=0.9\textwidth]{fermionloop}
          \end{figure}
          \begin{center}
            \alert{i.e., the "Hierarchy Problem"?}
          \end{center}
        \end{column}
      \end{columns}
    \end{column}
    \end{columns}
\end{frame}

% No-go theorems
%  - current understanding of QFT: characterised
%    by external spacetime symmetries and internal
%    (gauge) symmetries
%  - natural question: possible to describe by a
%    single unified symmetry group
%  - answer: no, due to CM (describe)
\begin{frame}
  \frametitle{The Coleman-Mandula Theorem}
\end{frame}

% Evading CM theorem
%  - can do so by violating assumption of only
%    bosonic generators, i.e., extend to graded
%    symmetry algebra
%  - HLS => SUSY is essentially unique such extension
%  - number of odd generators not restricted, but
%    focus on N = 1 SUSY (most relevant for TeV-scale
%    SUSY phenomenology)
\begin{frame}
  \frametitle{The HLS Theorem}
\end{frame}

% Super-Poincare algebra
%  - consequences: superpartners differ
%    in spin by 1/2, mass degeneracy
\begin{frame}
  \frametitle{The Super-Poincar\'{e} Algebra}
\end{frame}

\begin{frame}
  \frametitle{Implications of Global SUSY}
\end{frame}

% Local SUSY and supergravity
%  - very briefly, local SUSY => diffeomorphism invariance
%  - generalities of supergravity, but not details
\begin{frame}
  \frametitle{Local SUSY}
\end{frame}

% Hierarchy problem
%  - SUSY invariance => number of bosonic and fermionic
%    degrees of freedom must match (prove if space)
%  - in particular, all scalar degrees of freedom
%    associated with fermionic superpartners
%  - ensures scalar masses are symmetry protected from
%    quadratically divergent corrections (analogous to
%    chiral symmetry)
%  - relevance? SM contains a fundamental scalar, the Higgs,
%    soon realised that this should mix with heavy states
%    - known as the Hierarchy problem
%    - unbroken SUSY ``naturally'' cancels these divergent
%      corrections
\begin{frame}
  \frametitle{The Hierarchy Problem}
\end{frame}

% basic SUSY model building
\section{Basics of SUSY Model Building}

\begin{frame}
  \frametitle{Constructing SUSY Gauge Theories}
  \begin{block}{Toy Example: Wess-Zumino Model [1]}
    \begin{equation*}
      \mathcal{L}_{WZ} = \partial^\mu \phi^\dagger \partial_\mu \phi
      + i \bar{\psi} \bar{\sigma}^\mu \partial_\mu \psi + {\color{red}
        F^\dagger F} + \left [ -\frac{m}{2} \psi \psi - \frac{y}{2} \phi \psi
        \psi + \left ( L + m \phi + \frac{y}{2} \phi^2 \right ) {\color{red} F}
        + h.c. \right ]
    \end{equation*}
    Global SUSY transformation $\Rightarrow$
    $\delta \phi = \sqrt{2} \epsilon \psi$,
    $\delta \psi_\alpha = \sqrt{2} \epsilon_\alpha F - i \sqrt{2}
    ( \sigma^\mu \bar{\epsilon})_\alpha \partial_\mu \phi$,
    $\delta F = -i \sqrt{2} \bar{\epsilon} \bar{\sigma}^\mu \partial_\mu \psi$.
  \end{block}
  \begin{itemize}\itemsep1em
  \item \emph{In principle}, can construct any SUSY model in terms of
    ``component fields''
  \item But:
    \begin{itemize}\itemsep0.5em
    \item Transformation properties (e.g., of $\phi$, $\psi$, $F$ in WZ model)
      and SUSY invariance \alert{are not obvious}
    \item SUSY $\Rightarrow$ \alert{non-trivial constraints on field
      content, interaction terms}
    \item Close SUSY algebra on- and off-shell $\Rightarrow$ \alert{auxiliary
      fields} ({\color{red} $F$})
    \end{itemize}
  \item {\color{blue} Better: formulate theory in language in which
    SUSY invariance is manifest}
  \end{itemize}
  \vfill
      { \tiny [1] J.~Wess and B.~Zumino,
        \href{http://dx.doi.org/10.1016/0550-3213(74)90355-1}{%
          Nucl.~Phys.~B \textbf{70} (1974) 39.}}
\end{frame}

% Superspace formalism
%  - certainly introduce basics sufficient for N = 1 SUSY,
%    i.e., extend x -> (x, \theta, \bar{\theta})
%  - how technical to go?
%  - ordinary functions -> superfields, i.e., functions defined
%    on superspace
%  - general Lorentz scalar superfield can be written as a
%    terminating power series in the anticommuting variables
%  - (global) SUSY realised as rigid translations on superspace
\begin{frame}
  \frametitle{Superspace and Superfields}
  \begin{columns}[t]
    \begin{column}{0.5\textwidth}
      \begin{itemize}\itemsep1em
      \item {\color{blue} Natural and compact} technique for constructing
        SUSY models:
      \end{itemize}
      \vspace*{5pt}
      \begin{figure}
        \centering
        \includegraphics[width=0.7\textwidth]{%
          supergauge_transformations_title}
      \end{figure}
      \vspace*{-2pt}
      \begin{tikzpicture}
        \node[draw, fill = white, rounded corners = 6pt,
          blur shadow = {shadow blur steps = 5}]{%
          \includegraphics[width=\textwidth]{%
            supergauge_transformations_quote}
          };
      \end{tikzpicture}
    \end{column}
    \begin{column}{0.5\textwidth}
      \begin{itemize}\itemsep1em
      \item Minkowski space $\to$ ($d = 4$, $N = 1$) Minkowski superspace:
        \begin{gather*}
          x^\mu \to z^A \equiv
          (x^\mu, \theta^\alpha, \bar{\theta}_{\dot{\alpha}} ), \\
          \{ \theta^\alpha, \theta^\beta \} =
          \{ \bar{\theta}_{\dot{\alpha}}, \bar{\theta}_{\dot{\beta}} \} =
          \{ \theta^\alpha , \bar{\theta}_{\dot{\beta}} \} = 0
        \end{gather*}
      \item Elements of super-Poincar\'{e} group
        \begin{equation*}
          g = \exp \left [ i \left ( \alpha^\mu \hat{P}_\mu
            - \frac{1}{2} \omega^{\mu \nu} \hat{M}_{\mu \nu}
            + \epsilon \hat{Q} + \bar{\epsilon} \hat{\bar{Q}} \right )
            \right ]
        \end{equation*}
      \item Global/rigid SUSY transformations $\Rightarrow$ {  \color{blue}
        translations in superspace},
        \begin{gather*}
          x^\mu \to x^\mu - i \theta \sigma^\mu \bar{\epsilon}
          + i \epsilon \sigma^\mu \bar{\theta} , \\
          \theta^\alpha \to \theta^\alpha + \epsilon^\alpha , \quad
          \bar{\theta}_{\dot{\alpha}} \to \bar{\theta}_{\dot{\alpha}} +
          \bar{\epsilon}_{\dot{\alpha}}
        \end{gather*}
      \end{itemize}
    \end{column}
  \end{columns}
\end{frame}

% - mention R-symmetry anywhere?

% Representations of SUSY
%  - most general such superfield is reducible
%  - relevant for N = 1 model building: chiral
%    superfields,  give definition and supermultiplet
%    structure (including auxiliary fields)
%  - vector superfields, give definition and
%    supermultiplet structure
\begin{frame}
  \frametitle{Representations of SUSY}
  \begin{itemize}\itemsep1em
  \item Constructed in terms of superfields $\equiv$ functions on superspace,
    e.g.,
    \begin{equation*}
      \hat{S}(z) = a(x) + \theta \xi(x) + \bar{\theta} \bar{\chi}(x) +
      \theta \theta b(x) + \bar{\theta} \bar{\theta} c(x) + \bar{\theta}
      \bar{\sigma}^\mu \theta v_\mu (x)
      + \bar{\theta} \bar{\theta} \theta \zeta(x)
      + \theta \theta \bar{\theta} \bar{\lambda} (x) + \frac{1}{2} \theta
      \theta \bar{\theta} \bar{\theta} d(x)
    \end{equation*}
  \item Coefficients $a$, $b$, $\ldots$, $\xi$, $\bar{\chi}$, $\ldots$
    $\Rightarrow$ component fields of \emph{supermultiplet} (16 real d.o.f)
  \item Action of SUSY generators on (scalar) superfield
    \begin{gather*}
      \delta_\epsilon S = i \left [ \epsilon \hat{Q} + \bar{\epsilon}
        \hat{\bar{Q}}, S \right ]
      = i \left ( \epsilon Q + \bar{\epsilon} \bar{Q} \right ) S ,
      \\
      Q_\alpha = -i \left [ \frac{\partial}{\partial \theta^\alpha}
        + i \left ( \sigma^\mu \bar{\theta} \right )_\alpha
        \partial_\mu \right ] , \quad
      \bar{Q}^{\dot{\alpha}} = -i \left [
        \frac{\partial}{\partial \bar{\theta}_{\dot{\alpha}}} + i \left (
        \bar{\sigma}^\mu \theta \right )^{\dot{\alpha}} \partial_\mu \right ]
    \end{gather*}
    $\Rightarrow$ transformation laws for component fields
  \item General superfield is a \alert{reducible representation of SUSY}
  \end{itemize}
\end{frame}

\begin{frame}
  \frametitle{The Chiral Supermultiplet}
  \begin{itemize}\itemsep1em
  \item Constrain general superfield $\Rightarrow$ {\color{blue} irreducible
    representations of SUSY}
  \item Introduce chiral covariant derivatives:
    \begin{equation*}
      \mathcal{D}_\alpha = \frac{\partial}{\partial \theta^\alpha}
      - i \left (\sigma^\mu \bar{\theta} \right )_\alpha \partial_\mu , \quad
      \bar{\mathcal{D}}^{\dot{\alpha}} = \frac{\partial}
          {\partial \bar{\theta}_{\dot{\alpha}}}  - i \left ( \bar{\sigma}^\mu
            \theta \right )^{\dot{\alpha}} \partial_\mu
    \end{equation*}
  \item Impose $\bar{\mathcal{D}}_{\dot{\alpha}} \Phi = 0$ $\Rightarrow$ $\Phi
    \equiv$ {\color{blue} left chiral superfield} (similarly,
    $\mathcal{D}_\alpha \Phi^\dagger = 0$ $\Rightarrow$ right chiral
    superfield)
  \item Solve constraint $\Rightarrow$ obtain component field description
    of chiral supermultiplet
  \end{itemize}
  \begin{block}{Left chiral supermultiplet}
    \begin{equation*}
      \Phi = \phi(x) - i \theta\sigma^\mu\bar{\theta} \partial_\mu
      \phi(x) - \frac{1}{4} \theta\theta\bar{\theta}\bar{\theta} \partial^\mu
      \partial_\mu \phi(x) + \sqrt{2} \theta \psi(x)
      + \frac{i}{\sqrt{2}} \theta\theta
      \partial_\mu \psi(x)\sigma^\mu \bar{\theta} + \theta \theta F(x)
    \end{equation*}
    with component fields $\phi$, $\psi$, $F$ (i.e., 4 real scalar, 4
    real fermion d.o.f)
  \end{block}
\end{frame}

\begin{frame}
  \frametitle{The Vector Supermultiplet}
  \begin{itemize}
  \item Defined by condition $V = V^\dagger$ $\Rightarrow$
    \begin{align*}
      V(z) &= C(x) + \sqrt{2} \theta \xi(x) + \bar{\theta} \bar{\xi}(x)
      + \theta \theta M(x) + \bar{\theta} \bar{\theta} M^*(x) +
      \theta \sigma^\mu\bar{\theta}A_\mu(x) \\
      & \quad {} + \theta\theta \bar{\theta}
      \left [ \bar{\lambda}(x) - \frac{i}{\sqrt{2}} \bar{\sigma}^\mu\partial_\mu
        \xi(x) \right ]
        + \bar{\theta} \bar{\theta} \theta \left [ \lambda(x)
        - \frac{i}{\sqrt{2}} \sigma^\mu \partial_\mu \bar{\xi}(x) \right ]
      + \theta\theta \bar{\theta}\bar{\theta} \left [ \frac{1}{2} D(x) -
        \frac{1}{4} \partial^\mu\partial_\mu C(x) \right ]
    \end{align*}
  \item Extra auxiliary fields $C$, $M$, $\xi$ eliminated via supergauge
    transformation, e.g.,
    \begin{equation*}
      V \to V + i ( \Lambda - \Lambda^\dagger ) , \quad \text{ where } \Lambda
      = \text{chiral superfield}
    \end{equation*}
  \item NB still have ability to carry out ordinary gauge transformations
  \end{itemize}
  \begin{block}{Vector supermultiplet (WZ gauge)}
    \begin{equation*}
      V_{WZ}(z) = \theta \sigma^\mu \bar{\theta} A_\mu(x) + \theta \theta
      \bar{\theta} \bar{\lambda}(x) + \bar{\theta} \bar{\theta} \theta
      \lambda(x) + \frac{1}{2} \theta\theta\bar{\theta}\bar{\theta} D(x)
    \end{equation*}
    with component fields $A_\mu$, $\lambda$, $D$ (4 real scalar, 4 real
    fermion d.o.f.)
  \end{block}
\end{frame}

% SUSY Lagrangians
% - construction of globally SUSY invariant actions
% - introduce notions of integration on superspace
% - introduce definitions of F- and D-terms of superfield
% - interactions amongst chiral superfields characterised
%   by superpotential
% - usual kinetic terms etc. appropriately generalised to
%   be supergauge invariant
\begin{frame}
  \frametitle{Global SUSY Invariant Actions}
  \begin{itemize}\itemsep1em
  \item Component field transformation properties:
    \begin{equation*}
      \delta F = -i \sqrt{2} \bar{\epsilon} \bar{\sigma}^\mu \partial_\mu \psi ,
      \quad \delta D = {\color{red} \text{TO DO}}
    \end{equation*}
  \item $\theta \theta$ component of chiral supermultiplet
    can contribute to SUSY invariant action $\Rightarrow$ {\color{blue}
      ``$F$-term'' contribution}
  \item Similarly $\theta\theta\bar{\theta}\bar{\theta}$ component
    from vector supermultiplet $\Rightarrow$ {\color{blue} ``$D$-term''
      contribution}
  \item Berezin integral satisfying translation invariance and linearity:
    \begin{equation*}
      \int d\theta \, (f_0 + \theta f_1) = f_1 , \quad
      \int d^2 \theta \, \theta \theta = 1 , \quad
      \int d^2 \bar{\theta} \, \bar{\theta} \bar{\theta} = 1 \quad
      \left [ \text{note } \Rightarrow \frac{d f}{d\theta} = \int d\theta\,
        f \right ]
    \end{equation*}
  \item Then build action out of contributions of form
    ($d^4\theta \equiv d^2 \bar{\theta} d^2 \theta$,
    $\delta^{(2)}(\bar{\theta}) \equiv \bar{\theta} \bar{\theta}$ )
    \begin{equation*}
      \left [ V \right ]_D \equiv \int d^4\theta \, V , \quad
      \left [ \Phi \right ]_F \equiv \int d^4 \theta \, \delta^{(2)}(
      \bar{\theta}) \Phi
    \end{equation*}
  \end{itemize}
\end{frame}

% summarise standard form of invariant Lagrangian for
% non-Abelian SUSY gauge theory to which can refer
% back to
\begin{frame}
  \frametitle{Recipe for SUSY Models}
  \begin{block}{SUSY Gauge Theory}
    Given gauge group $G$ and chiral superfields $\Phi^I$ transforming
    under $G$,
    \begin{equation*}
      \mathcal{L} = \left [ \Phi^\dagger_I (e^V)^I{}_J \Phi^J \right ]_D
      + \frac{1}{16 k_i^2 g_i^2} \Tr \left [ \mathcal{W} \mathcal{W}
        + \bar{\mathcal{W}} \bar{\mathcal{W}} \right ]_F
      + \left [ W(\Phi^I) + h.c. \right ]_F
    \end{equation*}
  \end{block}
  \begin{itemize}
    \item Given gauge group $G$, chiral superfields
  \end{itemize}
\end{frame}

% SUSY breaking
% - empirically SUSY must be broken
% - broken SUSY still avoids quadratic divergences
%   provided softly broken
% - part of definining SUSY model => defining SUSY
%   breaking mechanism
% - explicit or spontaneous breaking,
%   ``top-down'' or ``bottom-up'' approach?
% - basic details of spontaneous SUSY breaking and
%   simplest examples, e.g. Planck mediated (permits
%   later introduction of CMSSM, CNMSSM, CE6SSM)
%     - only mention but do not give details of GMSB, AMSB etc.
% - ultimately, mechanism of high-scale SUSY breaking
%   (if it exists) remains unknown => for phenomenological
%   investigations write down (almost) most general
%   set of soft SUSY breaking interactions
%      - summarise, i.e., soft scalar masses, trilinears,
%        gaugino massses
\begin{frame}
  \frametitle{Soft SUSY Breaking}
\end{frame}

\begin{frame}
  \frametitle{Spontaneous SUSY Breaking}
\end{frame}

\begin{frame}
  \frametitle{Example: Gravity-mediated SUSY Breaking}
\end{frame}

\begin{frame}
  \frametitle{The Bottom-up Approach}
\end{frame}

% MSSM
\section{The MSSM}

% - general formalism now applied to construct phenomenologically
%   viable SUSY models
% - basic idea: ``supersymmetrise'' the SM by taking the known
%   matter content and gauge symmetries and embedding appropriately
%   into chiral and vector supermultiplets
% - simple considerations show that all superpartners must be
%   new states (e.g., no colour adjoint fermions observed)
% - thus SM -> MSSM by promoting all gauge fields to individual vector
%   supermultiplets, all matter fields to chiral superfields
% - one further complication: holomorphic superpotential and anomaly
%   cancellation => need second Higgs doublet chiral superfield
% - summarise field content
\begin{frame}
  \frametitle{Requirements on a Realistic SUSY Model}
\end{frame}

\begin{frame}
  \frametitle{The MSSM: A Schematic View}
\end{frame}

\begin{frame}
  \frametitle{MSSM Field Content}
\end{frame}

% - given matter content and symmetries, specification of SUSY
%   invariant part of MSSM completed by giving most general gauge
%   invariant and renormalisable superpotential, state
% - most general superpotential is dead on arrival due to too large
%   B and L violating interactions at tree-level
% - commonest scenario:rescue by imposing discrete R-parity
%   (discrete subgroup of U(1)_R symmetry, mention previously?)
% - structure of superpotential is reduced to only B and L conserving
%   interactions
\begin{frame}
  \frametitle{$R$-Parity in the MSSM}
\end{frame}

% - mention variants of different soft Lagrangian, e.g., pMSSM, CMSSM
\begin{frame}
  \frametitle{The Constrained MSSM}
\end{frame}

% phenomenological consequences of the MSSM
% - superpartner spectrum
\begin{frame}
  \frametitle{The Sparticle Spectrum}
\end{frame}

% - R-parity => natural DM candidate, the neutralino
\begin{frame}
  \frametitle{DM in the (RPC) MSSM}
\end{frame}

% - improved gauge unification, REWSB
\begin{frame}
  \frametitle{RG Evolution in the MSSM}
\end{frame}

% - as lead up to non-minimal models, current status?
%   - Update with most recent (2018?) LHC data?
%   - mention caveats: simplified model limits etc.
\begin{frame}
  \frametitle{Status and Prospects of the MSSM}
\end{frame}

% Non-minimal SUSY
\section{Non-minimal SUSY}

% - emphasise: often people equate SUSY = MSSM
% - indeed, know SM to be insufficient, so might expect MSSM also
%   to be limited
% - but no hard reason for SUSY to be minimal, can equally well
%   construct low-scale SUSY models with extended matter or
%   gauge sectors
% - these constitute non-minimal SUSY models
% - increasingly of relevance given the absence of evidence for
%   the MSSM
\begin{frame}
  \frametitle{SUSY $\not\equiv$ MSSM}
  \begin{center}
    The MSSM is \emph{a} realisation of TeV-scale SUSY, {\color{red}
      not the \emph{unique} such realisation}
  \end{center}
  \begin{itemize}\itemsep1em
  \item Ultimately, no fundamental reason for low-energy SUSY to be minimal
  \item MSSM shares some limitations of the minimal SM, e.g., $m_{\nu_i} = 0$
  \item SUSY $+$ extensions of SM at high-energies (e.g., RH neutrinos, GUTs)
    $\Rightarrow$ non-minimal SUSY models
  \item Minimal extension alone $\Rightarrow$ new puzzles/problems $\ldots$
    \begin{itemize}
      \item Best resolved in non-minimal models?
    \end{itemize}
  \end{itemize}
\end{frame}

% - examples of why you might want to extend the MSSM:
% - the little hierarchy problem: observed Higgs mass is difficult
%   to obtain in MSSM, requires heavy superpartners or large
%   mixings (pessimistic from the phenomenological point of view,
%   aesthetically somewhat against one of the common motivations
%   for SUSY, since implies large corrections to EW scale)
% - extend with new matter or gauge d.o.f. (=> extra F- or
%   D-terms in Lagrangian) raise Higgs mass
% - related: \mu problem, i.e., explain size of the \mu term
%   in the MSSM compared to soft terms
\begin{frame}
  \frametitle{The "Little Hierarchy Problem"}
   \begin{figure}
      \includegraphics[width=0.75\textwidth]{higgs_mass_prl}
    \end{figure}
    \begin{center}
      MSSM tree-level prediction:
    \begin{equation*}
      m_{h_1}^2 \leq m_Z^2 \cos^2 2\beta \lesssim (91 \text{ GeV})^2
   \end{equation*}
 \end{center}
\end{frame}

\begin{frame}
  \frametitle{The "Little Hierarchy Problem"}
  \begin{itemize}\itemsep1em
    \item $m_{h_1} \approx 125$ GeV $\Rightarrow$ large higher order
      corrections
      \begin{equation*}
        m_{h_1}^2 \approx m_Z^2 \cos^2 2\beta \left ( 1 - \frac{3}{8\pi^2}
          \frac{m_t^2}{v^2} \ln \frac{m_{\tilde{t}_1} m_{\tilde{t}_2}}{M_t^2}
          \right ) {+ \color{blue} \frac{3}{4\pi^2} \frac{m_t^4}{v^2} \ln
          \frac{m_{\tilde{t}_1} m_{\tilde{t}_2}}{M_t^2}} + \ldots
      \end{equation*}
    \item $\Rightarrow$ also large corrections to prediction for $m_Z$
      at SUSY scale $M_S$:
      \begin{equation*}
        \frac{m_Z^2}{2} = -\mu^2 + \frac{\overbrace{{\color{red} m_{H_d}^2} -
          {\color{red} m_{H_u}^2}}^{\text{RGE effects}} \tan^2\beta}
          {\tan^2\beta - 1} {\color{red} +  \delta_{1-\textrm{loop}}} ,
      \end{equation*}
      \begin{equation*}
          {\color{red}
          \delta_{1-\textrm{loop}} = \frac{3}{8\pi^2} \frac{m_t^2}{v^2
            \cos 2\beta} \left [ m_{\tilde{t}_1}^2 \left ( \ln
            \frac{m_{\tilde{t}_1}^2}{M_S^2} - 1 \right ) + m_{\tilde{t}_2}^2
            \left ( \ln \frac{m_{\tilde{t}_2}^2}{M_S^2} - 1 \right ) \right ]
            + \ldots }
      \end{equation*}
    \item $\Rightarrow$ naturalness problem?
    \item \alert{$\mu$-problem}: $m_Z^2/2 = -\mu^2 + \ldots \Rightarrow \mu
      \sim$ soft parameters?
  \end{itemize}
\end{frame}

% - mention shortcoming: give up minimality, what guides model-building?
% - obviously, non-minimal model must still reproduce SM results
% - otherwise extreme freedom: new matter fields, new gauge symmetries
% - guiding principle? E.g., some models can be seen to emerge from
%   UV complete theories (GUTs, strings, ...)
% - later can then mention Bayesian naturalness work as means of
%   discriminating between different models
\begin{frame}
  \frametitle{Going Beyond the MSSM}
\end{frame}

% simple example of non-minimal model doing this: the NMSSM
% - introduce NMSSM, simplest such extension where add a single
%   SM singlet superfield
% - positive features: larger Higgs mass, explains \mu problem
%   dynamically, improved prospects for e.g. baryogenesis
\begin{frame}
  \frametitle{Example: The NMSSM}
\end{frame}

% - no (concrete) evidence for MSSM or NMSSM, why would we consider then
%   a more complicated model?
% - can frame in context of Bayesian statistics, given the
%   observed limits can assess (given some set of priors)
%   plausibility of one versus the other
% - incorporates traditional, ad hoc fine tuning arguments
%   automatically
\begin{frame}
  \frametitle{Ockham's Razor}
  % Numquam ponenda est pluralitas sine necessitate
\end{frame}

% further reasons for extending the MSSM
% - improved gauge unification in the MSSM
%   => embed in a GUT model at high-energies?
% - motivates considerations of SUSY GUTs, e.g., SUSY SO(10),
%   E_6
% - emerge naturally in the context of superstring theory,
%   i.e., candidate theories of gravity, where breakdown of
%   SUSY in hidden sector is communicated to visible sector by
%   gravitational strength interactions (remind earlier discussion
%   of Planck mediated breaking)
% - rich phenomenology if exotic states survive to low energies
\begin{frame}
  \frametitle{Going Further}
\end{frame}

% example: the E6SSM and its variants
% - summarise some of the interesting features: general model
%   construction goes through as usual, matter embedded in complete
%   E6 reps => exotic coloured states
% - resolve \mu problem and avoid issues that come with scale-free
%   NMSSM, raise Higgs mass even further
% - briefly discuss some model building complications (e.g. Z2
%   symmetries), can resolve with E6 variants but do not go into
%   extensive detail
\begin{frame}
  \frametitle{$U(1)$ Extensions of the MSSM}
    \begin{equation*}
      SU(3)_C \times SU(2)_L \times U(1)_Y \to
      SU(3)_C \times SU(2)_L \times U(1)_Y \times U(1)^\prime
    \end{equation*}
    \vspace{-15pt}
  \begin{columns}[t]
    \begin{column}{0.45\textwidth}
      \begin{itemize} \itemsep1em
        \item Well-motivated extensions of MSSM, NMSSM
        \item $W_{\text{USSM}} \supset \lambda \hat{S} \hat{H}_d
          \cdot \hat{H}_u$, $Q^\prime_{\hat{S}} \neq 0$ $\Rightarrow$
          $\langle S \rangle = s / \sqrt{2}$ also breaks $U(1)^\prime$,
          generating {\color{blue} massive $Z^\prime$}
        \item Consistent model requires anomaly cancellation
          \begin{itemize}
            \item either family non-universal $U(1)^\prime$ charges
              or {\color{blue} extra matter}
          \end{itemize}
        \item Additional states $\Rightarrow$ exciting phenomenology
      \end{itemize}
    \end{column}
    \begin{column}{0.6\textwidth}
      \vspace{-10pt}
      \begin{figure}
        \centering
        \includegraphics[width=\textwidth]{treelevel_higgs_upperbound_plot}
      \end{figure}
    \end{column}
  \end{columns}
\end{frame}

\begin{frame}
  \frametitle{$E_6$ Inspired Models}
  \begin{itemize} \itemsep1em
     \item Lead to $U(1)$ extended models at low-energies:
      \begin{align*}
        E_6&\longrightarrow SO(10)\times U(1)_\psi \\
        &\longrightarrow SU(5)\times U(1)_\psi\times U(1)_\chi\\
        &\longrightarrow SU(3)_C\times SU(2)_L\times U(1)_Y\times
        U(1)_\psi\times U(1)_\chi\\
        &\longrightarrow SU(3)_C\times SU(2)_L\times U(1)_Y\times
        U(1)^\prime
      \end{align*}
    \item Resulting charges $Q' = Q_\chi \cos \theta_{E_6}
      + Q_\psi \sin \theta_{E_6}$, e.g., class of models
      \begin{table}[h]
        \begin{tabular}{cl}
          $U(1)_N$: & $Q_N = Q(\theta_{E_6} = \arctan\sqrt{15})$
            ($\equiv$ E$_6$SSM) \\
          $U(1)_\psi$: & $Q_\psi = Q(\theta_{E_6} = \pi / 2)$\\
          $U(1)_\eta$: & $Q_\eta = -Q(\theta_{E_6} = \pi -
            \arctan\sqrt{5/3})$\\
          $U(1)_I$: & $Q_I = -Q(\theta_{E_6} = \arctan\sqrt{3/5})$
        \end{tabular}
      \end{table}
    \item Matter content fills complete $\mathbf{27}$ representations
      ({\color{blue} ensures anomaly cancellation})
      \begin{itemize}
        \item $\Rightarrow$ additional exotic states
      \end{itemize}
  \end{itemize}
\end{frame}

\begin{frame}
  \frametitle{The E$_6$SSM}
    \begin{columns}[t]
      \begin{column}{0.5\textwidth}
        \begin{itemize} \itemsep0.2em
        \item $\tan\theta_{E_6} = \sqrt{15}$ $\Rightarrow$ $U(1)_N$
          under which right-handed neutrinos are uncharged
          \begin{itemize}
            \item allows {\color{blue} $\nu$ masses via see-saw} and
              {\color{blue} successful baryogenesis} [1]
          \end{itemize}
        \item Extra $\hat{L}_4$, $\hat{\overline{L}}_4$ from incomplete
          $\mathbf{27}^\prime$, $\mathbf{\overline{27}}^\prime$ for gauge
          unification
        \item Low-energy matter content from $\mathbf{27}$-plet:
          \begin{align*}
            &(\hat{Q}_i, \, \hat{u}^c_i, \, \hat{d}^c_i, \, \hat{L}_i, \,
            \hat{e}^c_i) + (\hat{D}_i, \, \hat{\overline{D}}_i)\\
            &\quad {} + (\hat{S}_{i}) + (\hat{H}^u_i) + (\hat{H}^d_i)
          \end{align*}
        \item Higgs doublets $\hat{H}^d_3$, $\hat{H}^u_3$ and one singlet
          $\hat{S}_3$ get VEVs ($\Rightarrow$ EWSB and break $U(1)_N$)
          \vfill
        \end{itemize}
      \end{column}
      \begin{column}{0.5\textwidth}
        \vspace{-40pt}
        \begin{table}[h]
          \centering
          \begin{tabular}{ccccc}
            \toprule
            & $SU(3)_C$ & $SU(2)_L$ & $\sqrt{\frac{5}{3}} Q_i^Y$
            & $\sqrt{40} Q_i^N$ \\
            \midrule
            $\hat{Q}_i$ & $\mathbf{3}$ & $\mathbf{2}$ & $\frac{1}{6}$ & $1$ \\
            $\hat{u}_i^c$ & $\mathbf{\overline{3}}$ & $\mathbf{1}$
            & $-\frac{2}{3}$ & $1$ \\
            $\hat{d}_i^c$ & $\mathbf{\overline{3}}$ & $\mathbf{1}$
            & $\frac{1}{3}$ & $2$ \\
            $\hat{L}_i$ & $\mathbf{1}$ & $\mathbf{2}$ & $-\frac{1}{2}$ & $2$ \\
            $\hat{e}_i^c$ & $\mathbf{1}$ & $\mathbf{1}$ & $1$ & $1$ \\
            $\hat{S}_i$ & $\mathbf{1}$ & $\mathbf{1}$ & $0$ & $5$ \\
            $\hat{H}_i^u$ & $\mathbf{1}$ & $\mathbf{2}$ & $\frac{1}{2}$
            & $-2$ \\
            $\hat{H}_i^d$ & $\mathbf{1}$ & $\mathbf{2}$ & $-\frac{1}{2}$
            & $-3$ \\
            $\hat{D}$ & $\mathbf{3}$ & $\mathbf{1}$ & $-\frac{1}{3}$ & $-2$ \\
            $\hat{\overline{D}}$ & $\mathbf{\overline{3}}$ &  $\mathbf{1}$
            & $\frac{1}{3}$ & $-3$ \\
            $\hat{L}_4$ & $\mathbf{1}$ & $\mathbf{2}$ & $-\frac{1}{2}$ & $2$ \\
            $\hat{\overline{L}}_4$ & $\mathbf{1}$ & $\mathbf{\overline{2}}$
            & $\frac{1}{2}$ & $-2$ \\
            \bottomrule
          \end{tabular}
        \end{table}
      \end{column}
    \end{columns}
    \vspace{-4pt}
    \begin{align*}
      \Aboxed{W_{\text{E}_6\text{SSM}} \approx y_{\tau} \hat{L}_3 \cdot
        \hat{H}^d_3 \hat{e}^c_3 + y_b \hat{Q}_3 \cdot \hat{H}^d_3 \hat{d}_3^c
        + y_t \hat{H}^u_3 \cdot \hat{Q}_3 \hat{u}_3^c + \lambda_i \hat{S}_3
        \hat{H}_i^d \cdot \hat{H}_i^u  + \kappa_i \hat{S}_3 \hat{D}_i
        \hat{\overline{D}}_i + \mu_L \hat{L}_4 \cdot \hat{\overline{L}}_4}
    \end{align*}
        {\tiny [1] S.~F.~King, S.~Moretti, and R.~Nevzorov,
          \href{http://dx.doi.org/10.1103/PhysRevD.73.035009}{Phys.~Rev.~D
            \textbf{73}, 035009 (2006)}
          (\href{http://arxiv.org/abs/hep-ph/0510419}{hep-ph/0510419})}
\end{frame}

\begin{frame}
  \frametitle{Experimental Signatures of the E$_6$SSM}
\end{frame}

\begin{frame}
  \frametitle{Limitations of Simple $E_6$ Models}
\end{frame}

\section{Summary}

\begin{frame}
  \frametitle{Summary}

  \begin{center}
    \large Thank you for listening!
  \end{center}

\end{frame}

\appendix

\begin{frame}
  \begin{center}
    {
      \Large
      Additional Slides
    }
  \end{center}
\end{frame}

% construction of Minkowski space as coset space of Poincare
% group factored by Lorentz group, analogy for superspace?

\begin{frame}
  \frametitle{Differentiation for Grassmann Variables}
\end{frame}

\end{document}
